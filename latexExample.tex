\documentclass[12pt]{article} % increase font size

\usepackage{listings}          % for creating language style
\input{arduinoLanguage.tex}    % adds the arduino language

%% Define an Arduino style fore use later %%
\lstdefinestyle{myArduino}{
  language=Arduino,
  %% Add other words needing highlighting below %%
  morekeywords=[1]{},                  % dark green
  morekeywords=[2]{FILE_WRITE},        % light blue
  morekeywords=[3]{SD, File},          % bold orange
  morekeywords=[4]{open, exists},      % orange
  %% The lines below add a nifty box around the code %%
  frame=shadowbox,                    
  rulesepcolor=\color{arduinoBlue},
}

\begin{document}

\section{SD Card Example}
Below is a snippet of arduino code.  It was added  via the input of the external
 file arduinoLanguage.tex.  A custom style was then created at the top of this
 file so that the non-built-in functions and classes like SD and open() would highlight
 properly.  The style also adds the pretty box around the code.\\

\begin{lstlisting}[style=myArduino]
#include "SoftwareSerial.h"
#include "TinyGPS.h"
#include "stdio.h"
#include "string.h"
#include <conio.h>

//GPS
TinyGPS gps;
SoftwareSerial ss(8, 7);  //8:RX,7:TX
SoftwareSerial Serial2(6, 5); //6:RX
//Cảm biến siêu âm
const int trig = 11;     // chân trig của HC-SR05
const int echo = 10;     // chân echo của HC-SR05
int timer;
//Module Sim A7
char aux_str[100];
const char * __APN      = "internet";
const char * __usrnm    = "";
const char * __password = "";

unsigned int Counter = 0;
unsigned long datalength, checksum, rLength;
unsigned short topiclength;
unsigned short topiclength2;
unsigned char topic[30];
char data[250];
char str[250];
unsigned char encodedByte;
int X;

unsigned short MQTTProtocolNameLength;
unsigned short MQTTClientIDLength;
unsigned short MQTTUsernameLength;
unsigned short MQTTPasswordLength;
unsigned short MQTTTopicLength;

const char * MQTTHost = "m23.cloudmqtt.com";    //IP của MQTT 
const char * MQTTPort = "12700";                //Port
const char * MQTTClientID = "ABCDEF";           //Client ID
const char * MQTTTopic = "data";                //Topic
const char * MQTTProtocolName = "MQIsdp";       //Protocol Name
const char MQTTLVL = 0x03;                      //Level
const char MQTTFlags = 0xC2;                    //Flags
const unsigned int MQTTKeepAlive = 60;          //Time Connect
const char * MQTTUsername = "fwenajrr";        
const char * MQTTPassword = "8Fzfn9sPb9CI";
const char MQTTQOS = 0x00;
const char MQTTPacketID = 0x0001;

//Khai báo biến toàn cục
int start = 0;
//GPS
float flat = 2.356825, flon=100.533558;    //Địa chỉ kinh độ và vĩ độ
unsigned long age, date, time, chars = 0;
int year;
byte month, day, hour, minute, second, hundredths;  //Thời gian
unsigned short sentences = 0, failed = 0;
//Cảm biến siêu âm
unsigned long duration; // biến đo thời gian
int New_distance,distance,const_distance;          // biến lưu khoảng cách
int delta;
//int pothole;


void setup()
{
  Serial.begin(9600);      //Giao tiep với module sim
  ss.begin(9600);          //Giao tiếp với NEO6 - GPS
  pinMode(trig,OUTPUT);    // chân trig sẽ phát tín hiệu
  pinMode(echo,INPUT);     // chân echo sẽ nhận tín hiệu
  start = 1;
  
  
  power_on();  //Hàm gửi liên tục lệnh AT để kiểm tra SIM800 có hoạt động ko, nếu có thì bỏ qua
  Serial.print("ATE0\r\n");
  delay(5000);
  Serial.print("AT+CIPMUX=0\r\n");
  delay(5000);
  Serial.print("AT+CSTT=\"internet\",\"\",\"\"\r\n");
  delay(5000);
  Serial.print("AT+CIICR\r\n");
  delay(5000);
  Serial.print("AT+CIFSR\r\n");  //Các lệnh cần thiết để thiết lập mạng cho sim
  delay(5000);

  
  New_distance = Detect();  //Lấy khoảng cách đầu mặt đường sau quá trình cài đặt
  location_get();           //Lấy toạ độ GPS sau quá trình cài đặt
  MQTTConnect();            //Kết nối tới server MQTT
  delay(2000);
}

void loop()
{
   if(start)  //Ban đầu gắn giá trị hằng số làm kc từ cảm biến tới mặt đường
   {
   const_distance = New_distance;
   }
   start = 0;
   distance = Detect();  //Đọc liên tục khoảng cách để check
   //delta = distance - const_distance;
   if((distance-const_distance)>=5)  //Nếu khoảng cách > 5cm ==> gửi toạ độ lên MQTT
   {
   location_get();
   MQTTConnect();  //Kết nối tới MQTT
   MQTTpublish();  //Gửi toạ độ lên MQTT
   //start = 1;
   }
   //delay(1000);*/
}


void power_on()
{
  uint8_t answer=0;
  int ctT=0;
 
  answer = sendATcommand("AT\r\n","OK",2000);
  delay(3000);
  while(answer==0)
  {
    answer = sendATcommand("AT\r\n","OK",2000);
    delay(3000);
    ctT++;
    if(ctT==10)  //Nếu quá 10 lần thì khởi động lại PWM Key
    {
      power_on();
    }
  }
 //sendATcommand("ATE0\r\n","OK",2000);
}

int8_t sendATcommand(char* ATcommand, char* expected_answer, unsigned int timeout)    //Hàm gửi tập lệnh AT và sẽ trả về giá trị là 1 nếu ko có lỗi khi gửi
{
  uint8_t x=0, answer=0;
  char response[50];
  unsigned long previous;
  char str;
  uint8_t index = 0;

  memset(response, '\0',50);
  delay(1000);
  //while( Serial.available() > 0) Serial.read();    // Clean the input buffer

  Serial.print(ATcommand);

  x = 0;
  previous = millis();    //Trả về là thời gian từ lúc ct bắt đầu hoạt động
  do
  {
    if(Serial.available() > 0)
    {    
      // if there is data in the UART input buffer, read it and checks for the asnwer
      response[x] = Serial.read();     //Read Response from Serial1 port
      //Serial.print(response[x]);        //Print response on Serial 0 port
      x++;

      // check if the desired answer  is in the response of the module
        if (strstr(response, expected_answer) != NULL)    
      {
        answer = 1;
      }
    }
  }//do
  // Waits for the asnwer with time out
  while((answer == 0) && ((millis() - previous) < timeout));    //Check till answer = 0 and timout period(ms)
  return answer;
}

/*
void SIM_Init()
{
  power_on();
  delay(2000);
  if(sendATcommand2("AT+CIPMUX=0\r\n","OK","ERROR",10000)==1)
  {
      while(sendATcommand("AT+CIPSTATUS\r\n","INITIAL", 2000)  == 0 );  // Check Current Connection Status
      delay(2000);
      if(sendATcommand2("AT+CSTT=\"internet\",\"\",\"\"\r\n","OK","ERROR",10000)==1)
      {
        while(sendATcommand("AT+CIPSTATUS\r\n", "START", 2000)  == 0 );
        delay(2000);
        if(sendATcommand2("AT+CIICR\r\n","OK","ERROR",8000)==1)
        {
          //delay(2000);
          while(sendATcommand("AT+CIPSTATUS\r\n", "GPRSACT", 2000)  == 0 );
          delay(2000);
          if(sendATcommand2("AT+CIFSR\r\n",".","ERROR",3000)==1);
          {
            while(sendATcommand("AT+CIPSTATUS", "GPRSACT", 2000)  == 0 );
            delay(3000);
              if (sendATcommand2("AT+CIPSTART=\"TCP\",\"m10.cloudmqtt.com\",14684\r\n", "CONNECT OK", "CONNECT FAIL", 5000) == 1);
            {
              /*sendATcommand2("AT+CIPSEND",">","ERROR",500);
              Serial.print("PUT /write_data.php?Date=20181997&Time130058&Latitude=2.0039&Longitude=100.23565");
              Serial.print(" HTTP/1.1\r\n");
              Serial.print("Host: potholemap.000webhostapp.com\r\n");
              Serial.print("Connection: keep-alive");
              Serial.write(0x0A);
              Serial.write(0x1A);
              //Serial.print("AT+HTTPPARA=\"URL\",\"http://potholemap.000webhostapp.com/write_data.php?Date=20181997&Time130058&Latitude=2.0039&Longitude=100.23565\"\r\n");
           
            }
          }
        }
      }
  }
}



int8_t sendATcommand2(String ATcommand, char* expected_answer1, char* expected_answer2, unsigned int timeout)
{
  uint8_t x=0, answer=0;
  char response[500];
  unsigned long previous;
  memset(response, '\0',500);
  delay(100);
  
  //while(Serial.available() > 0)  Serial1.read();
  
  Serial.println(ATcommand);
  x=0;
  previous =  millis();
  
  //this loops wait for the answer
  do
  {
    if(Serial.available() > 0)
    {
      response[x] = Serial.read();
      //Serial.print(response[x]);
      x++;
      //
      if(strstr(response, expected_answer1) != NULL){
        answer = 1;}
      else if(strstr(response, expected_answer2) != NULL){
        answer = 2;}
    }
  }
  while((answer == 0) && ((millis() - previous) < timeout));
  return answer;
}
*/
void MQTTConnect() {
  Serial.print("AT+CIPSTART=\"TCP\",\"m23.cloudmqtt.com\",12700\r\n");
  delay(7000);
  
  Serial.print("AT+CIPSEND\r\n");
  delay(1000);
    Serial.write(0x10);
    MQTTProtocolNameLength = strlen(MQTTProtocolName);
    MQTTClientIDLength = strlen(MQTTClientID);
    MQTTUsernameLength = strlen(MQTTUsername);
    MQTTPasswordLength = strlen(MQTTPassword);
    datalength = MQTTProtocolNameLength + 2 + 4 + MQTTClientIDLength + 2 + MQTTUsernameLength + 2 + MQTTPasswordLength + 2;
    X = datalength;
    do
    {
      encodedByte = X % 128;
      X = X / 128;
      // if there are more data to encode, set the top bit of this byte
      if ( X > 0 ) {
        encodedByte |= 128;
      }

      Serial.write(encodedByte);
    }
    while ( X > 0 );
    Serial.write(MQTTProtocolNameLength >> 8);
    Serial.write(MQTTProtocolNameLength & 0xFF);
    Serial.write(MQTTProtocolName);

    Serial.write(MQTTLVL); // LVL
    Serial.write(MQTTFlags); // Flags
    Serial.write(MQTTKeepAlive >> 8);
    Serial.write(MQTTKeepAlive & 0xFF);


    Serial.write(MQTTClientIDLength >> 8);
    Serial.write(MQTTClientIDLength & 0xFF);
    Serial.print(MQTTClientID);


    Serial.write(MQTTUsernameLength >> 8);
    Serial.write(MQTTUsernameLength & 0xFF);
    Serial.print(MQTTUsername);


    Serial.write(MQTTPasswordLength >> 8);
    Serial.write(MQTTPasswordLength & 0xFF);
    Serial.print(MQTTPassword);

    Serial.write(0x1A);
    
    
}


void MQTTpublish() {
  char kinhdodata[10],vidodata[10];
  char kinhdo[10],vido[10];


  
  Serial.print("AT+CIPSEND\r\n");
  delay(1000);

    memset(str, 0, sizeof(str));
    memset(data, 0, sizeof(data));
    
    
    location_get();
    dtostrf(flat, 3, 6, kinhdodata);
    dtostrf(flon, 3, 6, vidodata);
    sprintf((char*)data,"%d/%d/%d,%d:%d:%d,%s,%s",year,(int*)month,(int*)date,(int*)hour+7,(int*)minute,(int*)second,kinhdodata,vidodata);
   
    topiclength = sprintf((char*)topic, MQTTTopic);

    

    

    datalength = sprintf((char*)str, "%s%s", topic,data);

    delay(1000);
    Serial.write(0x30);
    X = datalength + 2;
    do
    {
      encodedByte = X % 128;
      X = X / 128;
      // if there are more data to encode, set the top bit of this byte
      if ( X > 0 ) {
        encodedByte |= 128;
      }
      Serial.write(encodedByte);
    }
    while ( X > 0 );

    Serial.write(topiclength >> 8);
    Serial.write(topiclength & 0xFF);
    Serial.print(str);
    Serial.write(0x1A);
    
    delay(1000);
    Serial.print("AT+CIPSHUT\r\n");
    delay(1000);
}

static void print_int(unsigned long val, unsigned long invalid, int len)
{
  char sz[32];
  if (val == invalid)
    strcpy(sz, "*******");
  else
    sprintf(sz, "%ld", val);
  sz[len] = 0;
  for (int i=strlen(sz); i<len; ++i)
    sz[i] = ' ';
  if (len > 0) 
    sz[len-1] = ' ';
  Serial.print(sz);
  smartdelay(0);
}
// Hàm in ngày tháng
// Hàm in chuỗi
static void print_str(const char *str, int len)
{
  int slen = strlen(str);
  for (int i=0; i<len; ++i)
    Serial.print(i<slen ? str[i] : ' ');
  smartdelay(0);
}





int Detect()
{
  /* Phát xung từ chân trig */
  digitalWrite(trig,0);   // tắt chân trig
  delayMicroseconds(2);
  digitalWrite(trig,1);   // phát xung từ chân trig
  delayMicroseconds(5);   // xung có độ dài 5 microSeconds
  digitalWrite(trig,0);   // tắt chân trig

  /* Tính toán thời gian */
  // Đo độ rộng xung HIGH ở chân echo. 
  duration = pulseIn(echo,HIGH);  
  // Tính khoảng cách đến vật.
  distance = int(duration/2/29.412);
  //pothole = digitalRead(echo);
  /*while(pothole == 0)
   {
   while(pothole == 1)
   {
   GPS_Send();
   break;
   }
   break;
   }*/
  delay(200);
  return distance;
}

static void location_get(void)
{
  //   float flat, flon;
  // unsigned long age, date, time, chars = 0;
  // unsigned short sentences = 0, failed = 0;
  static const double LONDON_LAT = 10.760497, LONDON_LON = 106.661788;	//thông tin tọa độ London
  // Lấy thông tin gps và in ra màn hình
  //print_int(gps.satellites(), TinyGPS::GPS_INVALID_SATELLITES, 5);
  //print_int(gps.hdop(), TinyGPS::GPS_INVALID_HDOP, 5);
  gps.f_get_position(&flat, &flon, &age);
  print_float(flat, TinyGPS::GPS_INVALID_F_ANGLE, 10, 6);
  print_float(flon, TinyGPS::GPS_INVALID_F_ANGLE, 11, 6);
  // print_int(age, TinyGPS::GPS_INVALID_AGE, 5);  //
  print_date(gps);
  //print_float(gps.f_altitude(), TinyGPS::GPS_INVALID_F_ALTITUDE, 7, 2);  //
  // print_float(gps.f_course(), TinyGPS::GPS_INVALID_F_ANGLE, 7, 2);  //
  //print_float(gps.f_speed_kmph(), TinyGPS::GPS_INVALID_F_SPEED, 6, 2);  //
  //print_str(gps.f_course() == TinyGPS::GPS_INVALID_F_ANGLE ? "*** " : TinyGPS::cardinal(gps.f_course()), 6);  //
  //print_int(flat == TinyGPS::GPS_INVALID_F_ANGLE ? 0xFFFFFFFF : (unsigned long)TinyGPS::distance_between(flat, flon, LONDON_LAT, LONDON_LON) / 1000, 0xFFFFFFFF, 9);  //        
  //print_float(flat == TinyGPS::GPS_INVALID_F_ANGLE ? TinyGPS::GPS_INVALID_F_ANGLE : TinyGPS::course_to(flat, flon, LONDON_LAT, LONDON_LON), TinyGPS::GPS_INVALID_F_ANGLE, 7, 2);  //
  //print_str(flat == TinyGPS::GPS_INVALID_F_ANGLE ? "*** " : TinyGPS::cardinal(TinyGPS::course_to(flat, flon, LONDON_LAT, LONDON_LON)), 6);  // 

  //  gps.stats(&chars, &sentences, &failed);  //
  //print_int(chars, 0xFFFFFFFF, 6);  //
  // print_int(sentences, 0xFFFFFFFF, 10);  //
  //print_int(failed, 0xFFFFFFFF, 9);  //
  Serial.println();

  smartdelay(1000);

}

static void smartdelay(unsigned long ms)
{
  unsigned long start = millis();
  do 
  {
    while (ss.available())
      gps.encode(ss.read());
  } 
  while (millis() - start < ms);
}

static void print_float(float val, float invalid, int len, int prec)
{
  if (val == invalid)
  {
    while (len-- > 1)
      Serial.print('*');
    Serial.print(' ');
  }
  else
  {
    Serial.print(val, prec);
    int vi = abs((int)val);
    int flen = prec + (val < 0.0 ? 2 : 1); // . and -
    flen += vi >= 1000 ? 4 : vi >= 100 ? 3 : vi >= 10 ? 2 : 1;
    for (int i=flen; i<len; ++i)
      Serial.print(' ');
  }
  smartdelay(0);
}

static void print_date(TinyGPS &gps)
{
  unsigned long age;
  gps.crack_datetime(&year, &month, &day, &hour, &minute, &second, &hundredths, &age);
  if (age == TinyGPS::GPS_INVALID_AGE)
    Serial.print("********** ******** ");
  else
  {
    char sz[32];
    sprintf(sz, "%02d/%02d/%02d %02d:%02d:%02d ",
    month, day, year, hour+7, minute, second);
    Serial.print(sz);
  }
  smartdelay(0);
}

\end{lstlisting}


\end{document}