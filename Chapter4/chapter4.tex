\headerandfooterconfig

\chapter{Thiết kế và thực hiện phần mềm}

\section{Thiết kế kiến trúc phần mềm}
\subsection{Các thành phần}
Kiến trúc phần mềm của hệ thống gồm nhiều thành phần:
\begin{itemize}
	\item Đọc và xử lý data từ các Tags.
	\item Tạo Server có kết nối với CSDL MongoDB để lưu data.
	\item Tạo Web service hiển thị trực quan các thông tin từ CSDL MongoDB.
\end{itemize}
\subsection{Yêu cầu thiết kế}
\begin{itemize}
	\item Đáp ứng thực hiện đầy đủ các thành phần trong kiến trúc phần mềm của hệ thống.
	\item Đảm bảo hệ thống vận hành chính xác, hiệu quả, độ trễ thấp, kết nối bền vững.
	\item Giao diện web thân thiện, dễ truy cập, ít tốn tài nguyên.
\end{itemize}

\section{Thực hiện phần mềm}
\subsection{Cài đặt và cấu hình MongoDB}
\subsubsection{Cài đặt MongoBD}
Hệ điều hành sử dụng trong đề cương: Ubuntu 18.04.1 LTS.\\
Tiến hành truy cập trang chủ $https://docs.mongodb.com/manual/tutorial/install-mongodb-on-ubuntu/$ và cài đặt theo các hướng dẫn.
\Figure{htp}{12}{Chapter4/Chapter4Figs/fig4_1}{Hướng dẫn cài đặt MongoDB trên Ubuntu}
\label{ref{fig4_1}}
\subsubsection{Cấu hình MongoDB}
Mở Terminal và gõ \textbf{sudo service mongod start} để khởi chạy MongoDB.\\
\Figure{htp}{12}{Chapter4/Chapter4Figs/fig4_2}{Khởi chạy MongoDB}
\label{ref{fig4_2}}

Gõ \textbf{mongo} để truy xuất MongoDB.
\Figure{htp}{12}{Chapter4/Chapter4Figs/fig4_3}{Truy xuất MongoDB}
\label{ref{fig4_3}}
Gõ \textbf{show dbs} để xem tất cả các database trong máy.
\Figure{htp}{12}{Chapter4/Chapter4Figs/fig4_4}{Xem tất cả các database MongoDB}
\label{ref{fig4_4}}

\subsection{Tạo Server truy xuất MongoDB}
Ngôn ngữ lập trình NodeJS được sử dụng. Nhiệm vụ của server là kết nối với MongoDB để truy xuất dữ liệu bằng các phương thức GET/POST của giao thức HTTP.

\subsubsection{Cài đặt NodeJS và Tạo Project}
Yêu cầu: hệ điều hành đã cài đặt NodeJS.\\
Tiến hành tạo thư mục chứa project có tên \textbf{nodejs-mongodb}.\\
Mở Terminal và gõ \textbf{npm init} để tạo project nodejs.\\
\Figure{htp}{12}{Chapter4/Chapter4Figs/fig4_5}{Tạo project nodejs}
\label{ref{fig4_5}}

\subsubsection{Cài đặt module mongoose}
Mongoose là một thư viện mô hình hóa đối tượng (Object Data Model - ODM) cho MongoDB và Node.js. Nó quản lý mối quan hệ giữa dữ liệu, cung cấp sự xác nhận giản đồ và được sử dụng để dịch giữa các đối tượng trong mã và biểu diễn các đối tượng trong MongoDB.
\Figure{htp}{12}{Chapter4/Chapter4Figs/fig4_6_1}{Mô hình mongoose, nodejs, mongodb}
\label{ref{fig4_6_1}}
Gõ \textbf{npm install --save mongoose}
\Figure{htp}{12}{Chapter4/Chapter4Figs/fig4_6}{Cài đặt module mongoose}
\label{ref{fig4_6}}

\subsubsection{Cài đặt module express}
Express framework được cài đặt và được sử dụng để tạo một máy chủ web.
Gõ \textbf{npm install --save express}
\Figure{htp}{12}{Chapter4/Chapter4Figs/fig4_7}{Cài đặt module express}
\label{ref{fig4_7}}

\subsubsection{Tạo web server dùng express}
Trong thư mục project tạo file \textbf{app.js} 
\begin{lstlisting}
var express = require('express'), //Import the express module

  app = express(),

  port = process.env.PORT || 8000,

  mongoose = require('mongoose'), //Import the mongoose module

  Tag = require('./restapi/models/tagModel'),

  bodyParser = require('body-parser'); //Import the body-parser module
  
// Get Mongoose to use the global promise library
mongoose.Promise = global.Promise;

//Set up default mongoose connection
mongoose.connect('mongodb://localhost/rfid');

app.use(bodyParser.urlencoded({ extended: true }));

app.use(bodyParser.json());

var routes = require('./restapi/routes/tagRoutes');

routes(app);

app.use(function(req, res) {

  res.status(404).send({url: req.originalUrl + ' not found'})

});

app.listen(port);

console.log('rfid -  RESTful web services with Nodejs started on: ' + port);
\end{lstlisting}
Tạo thư mục \textbf{restapi}. Trong thư mục \textbf{restapi} taoj các thư mục con \textbf{controller}, \textbf{models} và \textbf{routes}.
\subsubsection{Tạo file Controller, Model và Routes}
Trong thư mục \textbf{models} tạo file  \textbf{tagModel.js} để tạo các Schema cho tag.
\begin{lstlisting}
'use strict';

var mongoose = require('mongoose'); //Require Mongoose


//Define a schema
var Schema = mongoose.Schema;
 
var TagSchema = new Schema({

	Ant_Id: { type: String, required:  'Ant_Id cannot be left blank.' },

	PC: { type: String, required:  'PC cannot be left blank.' },
  
	EPC:    { type: String, required:  'EPC cannot be left blank.', index: {unique: true}},

	Date:  {type: Date, required: true, default: Date.now }
});
// Compile model from schema
module.exports = mongoose.model('Tags', TagSchema);

//Define a schema
var LogSchema = new Schema({

	Ant_Id: { type: String, required:  'Ant_Id cannot be left blank.' },

	PC: { type: String, required:  'PC cannot be left blank.' },
  
	EPC:    { type: String, required:  'EPC cannot be left blank.'},

	Date:  {type: Date, required: true, default: Date.now }
});
// Compile model from schema
module.exports = mongoose.model('Logs', LogSchema);
\end{lstlisting}

Trong đó, AntID, PC, EPC, Date là những thông tin cần có trong tag database.\\

Trong thư mục \textbf{controller} tạo file  \textbf{tagController.js} để quản lý các phương thức truy xuất database.
\begin{lstlisting}
'use strict';

var mongoose = require('mongoose');

var Tag = mongoose.model('Tags');

var Log = mongoose.model('Logs');

exports.tags = function(req, res) {
  Tag.find({}, function(err, tag) {
    if (err)
      res.send(err);
    res.json(tag);
  });
};

exports.add = function(req, res) {
  var new_tag = new Tag(req.body);
  new_tag.save(function(err, tag) {
    if (err)
      res.send(err);
    res.json(tag);
  });
};

exports.gettag = function(req, res) {
  Tag.findById(mongoose.Types.ObjectId(req.query.tagId), function(err, tag) {
    if (err)
      res.send(err);
    res.json(tag);
  });
};

exports.update = function(req, res) {
  var id = mongoose.Types.ObjectId(req.query.tagId);
  Tag.findOneAndUpdate({_id: id}, req.body, {new: true}, function(err, tag) {
    if (err)
      res.send(err);
    res.json(tag);
  });
};

exports.delete = function(req, res) {
  var id = mongoose.Types.ObjectId(req.query.tagId);
  Tag.remove({
    _id: id
  }, function(err, tag) {
    if (err)
      res.send(err);
    res.json({ message: 'Tag deleted successfully' });
  });
};


//////for logs

    
exports.logs = function(req, res) {
  Log.find({}, function(err, log) {
    if (err)
      res.send(err);
    res.json(log);
  });
};

exports.add_log = function(req, res) {
  var new_log = new Log(req.body);
  new_log.save(function(err, log) {
    if (err)
      res.send(err);
    res.json(log);
  });
};

exports.getlog = function(req, res) {
  Log.findById(mongoose.Types.ObjectId(req.query.logId), function(err, log) {
    if (err)
      res.send(err);
    res.json(log);
  });
};

exports.update_log = function(req, res) {
  var log_id = mongoose.Types.ObjectId(req.query.logId);
  Log.findOneAndUpdate({_id: log_id}, req.body, {new: true}, function(err, log) {
    if (err)
      res.send(err);
    res.json(log);
  });
};

exports.delete_log = function(req, res) {
  var log_id = mongoose.Types.ObjectId(req.query.logId);
  Log.remove({
    _id: log_id
  }, function(err, log) {
    if (err)
      res.send(err);
    res.json({ message: 'Tag deleted successfully' });
  });
};
\end{lstlisting}

Trong thư mục \textbf{routes} tạo file  \textbf{tagRoutes.js} để quản lý các đường dẫn dùng để truy xuất database.
\begin{lstlisting}
'use strict';

module.exports = function(app) {

    var tag = require('../controllers/tagController');

    var log = require('../controllers/tagController');

    app.route('/tags')

        .get(tag.tags)

        .post(tag.add);

    app.route('/tags/:tagId')

        .get(tag.gettag)

        .put(tag.update)
        
        .delete(tag.delete);

    app.route('/logs')

        .get(log.logs)

        .post(log.add_log);

    app.route('/logs/:logId')

        .get(log.getlog)

        .put(log.update_log)
        
        .delete(log.delete_log);
};
\end{lstlisting}
Trong thư mục chứa project, mở Terminal và gõ \textbf{node app.js} để khởi chạy ứng dụng.
\Figure{htp}{12}{Chapter4/Chapter4Figs/fig4_8}{Khởi chạy ứng dụng NodeJS}
\label{ref{fig4_8}}

\subsection{Tạo ứng dụng web hiển thị trực quan MongoDB}
Module được sử dụng \textbf{express-generator}.\\
Tiến hành update Express và Express-generator, mở terminal và gõ \textbf{npm update -g express}.\\
Gõ tiếp \textbf{npm update -g express-generator}.\\
Cuối cùng, gõ \textbf{express uhf-rfid-web} để tạo project express.
Trong file \textbf{package.json} chỉnh sửa để thêm một số module cần cho MongoDB.
\Figure{htp}{12}{Chapter4/Chapter4Figs/fig4_9}{File package.json}
\label{ref{fig4_9}}

Mở thư mục \textbf{views},bắt đầu với file \textbf{layout.jade}
\begin{lstlisting}
doctype html
html
  head
    title= title
    meta(name='viewport', content='width=device-width, initial-scale=1')
    link(rel='shortcut icon', type='image/png' href='/images/database.png')
    link(rel='stylesheet', href='/stylesheets/style.css')
  body
    block content
    script(src='http://ajax.googleapis.com/ajax/libs/jquery/2.1.1/jquery.min.js')
    script(src='/javascripts/global.js')
\end{lstlisting}

Mở \textbf{index.js} để tạo web chính
\begin{lstlisting}
extends layout

block content
  h1= title
  h3 Welcome to our test

  // Wrapper
  #wrapper

    // TAG INFO
    #tagInfo
      h2 Tag Info
      p
        strong Id:
        |  <span id='tagInfoId'></span>
        br
        strong AntennaId:
        |  <span id='tagInfoAntennaId'></span>
        br
        strong PC:
        |  <span id='tagInfoPc'></span>
        br
        strong EPC:
        |  <span id='tagInfoEpc'></span>
        br
        strong Date:
        |  <span id='tagInfoDate'></span>
    // /TAG INFO

    // TAG LIST
    

    #tagList
      h2 Tag List
      table
        thead
          th ID
          // th ANTENNA ID
          // th PC 
          th EPC
          // th DATE
          th DELETE?
        tbody
    // /TAG LIST

  // /FOOTER
  // #footer
  //    h5 This project  
  // /WRAPPER
\end{lstlisting}

Mở app.js và chỉnh sửa như sau, app.js sẽ vận hành tất cả các thao tác để tạo web hiển thị thông tin từ database cho người dùng.
\begin{lstlisting}
var createError = require('http-errors');
var express = require('express');
var path = require('path');
var cookieParser = require('cookie-parser');
var logger = require('morgan');

// Database
var mongo = require('mongodb');
var monk = require('monk');
var db = monk('localhost:27017/rfid');
var indexRouter = require('./routes/index');
var tagsRouter = require('./routes/tags');
var logsRouter = require('./routes/logs');

var app = express();

// view engine setup
app.set('views', path.join(__dirname, 'views'));
app.set('view engine', 'jade');

app.use(logger('dev'));
app.use(express.json());
app.use(express.urlencoded({ extended: false }));
app.use(cookieParser());
app.use(express.static(path.join(__dirname, 'public')));

// Make our db accessible to our router
app.use(function(req,res,next){
    req.db = db;
    next();
});

app.use('/', indexRouter);
app.use('/tags', tagsRouter);
app.use('/logs', logsRouter);

// catch 404 and forward to error handler
app.use(function(req, res, next) {
  next(createError(404));
});

// error handler
app.use(function(err, req, res, next) {
  // set locals, only providing error in development
  res.locals.message = err.message;
  res.locals.error = req.app.get('env') === 'development' ? err : {};

  // render the error page
  res.status(err.status || 500);
  res.render('error');
});

module.exports = app;
\end{lstlisting}

Trong thư mục \textbf{routes}, mở file \textbf{index.js}, \textbf{index.js} sẽ chứa đường dẫn truy cập vào web hiển thị thông tin.
\begin{lstlisting}
var express = require('express');
var router = express.Router();

/* GET home page. */
router.get('/', function(req, res, next) {
  res.render('index', { title: 'UHF RFID' });
});
router.get('/logs', function(req, res, next) {
	res.render('index_logs', { title: 'UHF RFID LOGS' });
});

module.exports = router;

\end{lstlisting}

Trong thư mục public/javascripts, tạo file global.js. Trong đây sẽ chứa các function truy cập database và truy xuất dữ liệu.
\begin{lstlisting}
// Taglist data array for filling in info box
var tagListData = [];

// DOM Ready =============================================================
$(document).ready(function() {

  // Populate the tag table on initial page load
  populateTable();
  // Tag link click
$('#tagList table tbody').on('click', 'td a.linkshowtag', showTagInfo);
});
  // Add User button click
  $('#btnAddTag').on('click', addTag);
    // Delete User link click
  $('#tagList table tbody').on('click', 'td a.linkdeletetag', deleteTag);

// Functions =============================================================

// Fill table with data
function populateTable() {

  // Empty content string
  var tableContent = '';


  // jQuery AJAX call for JSONvar express = require('express');
var router = express.Router();

/* GET home page. */
router.get('/', function(req, res, next) {
  res.render('index', { title: 'UHF RFID' });
});
router.get('/logs', function(req, res, next) {
	res.render('index_logs', { title: 'UHF RFID LOGS' });
});

module.exports = router;
  $.getJSON( '/tags/taglist', function( data ) {
    // Stick our user data array into a taglist variable in the global object
tagListData = data;
    // For each item in our JSON, add a table row and cells to the content string
    $.each(data, function(){
      tableContent += '<tr>';
      tableContent += '<td><a href="#" class="linkshowtag" rel="' + this._id + '">' + this._id + '</a></td>';
      // tableContent += '<td>' + this.Ant_Id + '</td>';
      // tableContent += '<td>' + this.PC + '</td>';
      tableContent += '<td>' + this.EPC + '</td>';
      // tableContent += '<td>' + this.Date + '</td>';
      tableContent += '<td><a href="#" class="linkdeletetag" rel="' + this._id + '">DELETE</a></td>';
      tableContent += '</tr>';
    });

    // Inject the whole content string into our existing HTML table
    $('#tagList table tbody').html(tableContent);
  });
};

// Show Tag Info
function showTagInfo(event) {

  // Prevent Link from Firing
  event.preventDefault();

  // Retrieve username from link rel attribute
  var thisTagName = $(this).attr('rel');

  // Get Index of object based on id value
  var arrayPosition = tagListData.map(function(arrayItem) { return arrayItem._id; }).indexOf(thisTagName);
  // Get our Tag Object
  var thisTagObject = tagListData[arrayPosition];
  //Handle Date Data
  var date = new Date(thisTagObject.Date);
  var a = date.toDateString() +'  '+date.toTimeString();

  //Populate Info Box
  $('#tagInfoId').text(thisTagObject._id);
  $('#tagInfoAntennaId').text(thisTagObject.Ant_Id);
  $('#tagInfoPc').text(thisTagObject.PC);
  $('#tagInfoEpc').text(thisTagObject.EPC);
  $('#tagInfoDate').text(a);
  // $('#tagInfoDate').text(thisTagObject.Date);
};

// Add Tag
function addTag(event) {
  event.preventDefault();

  // Super basic validation - increase errorCount variable if any fields are blank
  var errorCount = 0;
  $('#addTag input').each(function(index, val) {
    if($(this).val() === '') { errorCount++; }
  });

  // Check and make sure errorCount's still at zero
  if(errorCount === 0) {

    // If it is, compile all user info into one object
    var newTag = {
      'Ant_Id': $('#addTag fieldset input#inputTagAntennaId').val(),
      'PC': $('#addTag fieldset input#inputTagPC').val(),
      'EPC': $('#addTag fieldset input#inputTagEPC').val(),
      }

    // Use AJAX to post the object to our adduser service
    $.ajax({
      type: 'POST',
      data: newTag,
      url: '/tags/addtag',
      dataType: 'JSON'
    }).done(function( response ) {

      // Check for successful (blank) response
      if (response.msg === '') {

        // Clear the form inputs
        $('#addUser fieldset input').val('');

        // Update the table
        populateTable();

      }
      else {

        // If something goes wrong, alert the error message that our service returned
        alert('Error: ' + response.msg);

      }
    });
  }
  else {
    // If errorCount is more than 0, error out
    alert('Please fill in all fields');
    return false;
  }
};

// Delete Tag
function deleteTag(event) {

  event.preventDefault();

  // Pop up a confirmation dialog
  var confirmation = confirm('Are you sure you want to delete this tag?');

  // Check and make sure the user confirmed
  if (confirmation === true) {

    // If they did, do our delete
    $.ajax({
      type: 'DELETE',
      url: '/tags/deletetag/' + $(this).attr('rel')
    }).done(function( response ) {

      // Check for a successful (blank) response
      if (response.msg === '') {
      }
      else {
        alert('Error: ' + response.msg);
      }

      // Update the table
      populateTable();

    });

  }
  else {

    // If they said no to the confirm, do nothing
    return false;

  }

};

\end{lstlisting}


Để chạy ứng dụng web, mở Terminal và chạy \textbf{npm start}
\Figure{htp}{12}{Chapter4/Chapter4Figs/fig4_10}{Chạy ứng dụng web}
\label{ref{fig4_10}}



