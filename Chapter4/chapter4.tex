\headerandfooterconfig

\chapter{Thiết kế và thực hiện phần mềm}

\section{Thiết kế kiến trúc phần mềm}
\subsection{Các thành phần}
Kiến trúc phần mềm của hệ thống gồm nhiều thành phần:
\begin{itemize}
	\item Đọc và xử lý data từ các Tags.
	\item Tạo Server có kết nối với CSDL MongoDB để lưu data.
	\item Tạo Web service hiển thị trực quan các thông tin từ CSDL MongoDB.
\end{itemize}
\subsection{Yêu cầu thiết kế}
\begin{itemize}
	\item Đáp ứng thực hiện đầy đủ các thành phần trong kiến trúc phần mềm của hệ thống.
	\item Đảm bảo hệ thống vận hành chính xác, hiệu quả, độ trễ thấp, kết nối bền vững.
	\item Giao diện web thân thiện, dễ truy cập, ít tốn tài nguyên.
\end{itemize}
\subsection{Lưu đồ giải thuật}
\Figure{htp}{14}{Chapter4/Chapter4Figs/fig4_11}{Lưu đồ giải thuật}
\label{ref{fig4_11}}
\section{Thực hiện phần mềm}
\subsection{Đọc và xử lý dữ liệu trong tag}
\begin{itemize}
	\item Bước 1: Gửi tín hiệu yêu cầu đến Module UHF RFID
	\item Bước 2: Nhận các EPC ứng với các tag đang ở trong phạm vi dựa vào chuẩn giao tiếp
	\item Bước 3: Kiểm tra thông tin hiện có của EPC trong Database: Nếu thông tin không hợp lệ, gửi ra mã lỗi trên Terminal và không cập nhật trên Database; nếu thông tin hợp lệ, cập nhật trạng thái mới nhất của tag vào Database
\end{itemize}
\subsection{Cài đặt và cấu hình MongoDB}
\subsubsection{Cài đặt MongoDB}
Hệ điều hành sử dụng trong đề cương: Ubuntu 18.04.1 LTS.\\
Tiến hành truy cập trang chủ $https://docs.mongodb.com/manual/tutorial/install-mongodb-on-ubuntu/$ và cài đặt theo các hướng dẫn.
\Figure{htp}{12}{Chapter4/Chapter4Figs/fig4_1}{Hướng dẫn cài đặt MongoDB trên Ubuntu}
\label{ref{fig4_1}}
\subsubsection{Cấu hình MongoDB}
Mở Terminal và gõ \textbf{sudo service mongod start} để khởi chạy MongoDB.\\
\Figure{htp}{12}{Chapter4/Chapter4Figs/fig4_2}{Khởi chạy MongoDB}
\label{ref{fig4_2}}

Gõ \textbf{mongo} để truy xuất MongoDB.
\Figure{htp}{12}{Chapter4/Chapter4Figs/fig4_3}{Truy xuất MongoDB}
\label{ref{fig4_3}}
Gõ \textbf{show dbs} để xem tất cả các database trong máy.
\Figure{htp}{12}{Chapter4/Chapter4Figs/fig4_4}{Xem tất cả các database MongoDB}
\label{ref{fig4_4}}

\subsection{Tạo Server truy xuất MongoDB}
Ngôn ngữ lập trình NodeJS được sử dụng. Nhiệm vụ của server là kết nối với MongoDB để truy xuất dữ liệu bằng các phương thức GET/POST của giao thức HTTP.

\subsubsection{Cài đặt NodeJS và Tạo Project}
Yêu cầu: hệ điều hành đã cài đặt NodeJS.\\
Tiến hành tạo thư mục chứa project có tên \textbf{nodejs-mongodb}.\\
Mở Terminal và gõ \textbf{npm init} để tạo project nodejs.\\
\Figure{htp}{12}{Chapter4/Chapter4Figs/fig4_5}{Tạo project nodejs}
\label{ref{fig4_5}}

\subsubsection{Cài đặt module mongoose}
Mongoose là một thư viện mô hình hóa đối tượng (Object Data Model - ODM) cho MongoDB và Node.js. Nó quản lý mối quan hệ giữa dữ liệu, cung cấp sự xác nhận giản đồ và được sử dụng để dịch giữa các đối tượng trong mã và biểu diễn các đối tượng trong MongoDB.
\Figure{htp}{12}{Chapter4/Chapter4Figs/fig4_6_1}{Mô hình mongoose, nodejs, mongodb}
\label{ref{fig4_6_1}}
Gõ \textbf{npm install --save mongoose}
\Figure{htp}{12}{Chapter4/Chapter4Figs/fig4_6}{Cài đặt module mongoose}
\label{ref{fig4_6}}

\subsubsection{Cài đặt module express}
Express framework được cài đặt và được sử dụng để tạo một máy chủ web.
Gõ \textbf{npm install --save express}
\Figure{htp}{12}{Chapter4/Chapter4Figs/fig4_7}{Cài đặt module express}
\label{ref{fig4_7}}

\subsubsection{Tạo web server dùng express}
Trong thư mục project tạo file \textbf{app.js} 

Tạo thư mục \textbf{restapi}. Trong thư mục \textbf{restapi} taoj các thư mục con \textbf{controller}, \textbf{models} và \textbf{routes}.
\subsubsection{Tạo file Controller, Model và Routes}
Trong thư mục \textbf{models} tạo file  \textbf{tagModel.js} để tạo các Schema cho tag.

Tạo thư mục \textbf{restapi}. Trong thư mục \textbf{restapi} taoj các thư mục con \textbf{controller}, \textbf{models} và \textbf{routes}.
\subsubsection{Tạo file Controller, Model và Routes}
Trong thư mục \textbf{models} tạo file  \textbf{tagModel.js} để tạo các Schema cho tag.


Trong đó, AntID, PC, EPC, Date là những thông tin cần có trong tag database.\\

Trong thư mục \textbf{controller} tạo file  \textbf{tagController.js} để quản lý các phương thức truy xuất database.

Trong thư mục chứa project, mở Terminal và gõ \textbf{node app.js} để khởi chạy ứng dụng.
\Figure{htp}{12}{Chapter4/Chapter4Figs/fig4_8}{Khởi chạy ứng dụng NodeJS}
\label{ref{fig4_8}}

\subsection{Tạo ứng dụng web hiển thị trực quan MongoDB}
Module được sử dụng \textbf{express-generator}.\\
Tiến hành update Express và Express-generator, mở terminal và gõ \textbf{npm update -g express}.\\
Gõ tiếp \textbf{npm update -g express-generator}.\\
Cuối cùng, gõ \textbf{express uhf-rfid-web} để tạo project express.
Trong file \textbf{package.json} chỉnh sửa để thêm một số module cần cho MongoDB.
\Figure{htp}{12}{Chapter4/Chapter4Figs/fig4_9}{File package.json}
\label{ref{fig4_9}}

Mở thư mục \textbf{views},bắt đầu với file \textbf{layout.jade}


Trong thư mục \textbf{routes}, mở file \textbf{index.js}, \textbf{index.js} sẽ chứa đường dẫn truy cập vào web hiển thị thông tin.


Trong thư mục public/javascripts, tạo file global.js. Trong đây sẽ chứa các function truy cập database và truy xuất dữ liệu.


Để chạy ứng dụng web, mở Terminal và chạy \textbf{npm start}
\Figure{htp}{12}{Chapter4/Chapter4Figs/fig4_10}{Chạy ứng dụng web}
\label{ref{fig4_10}}



