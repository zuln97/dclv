\cleardoublepage
\addcontentsline{toc}{chapter}{Tóm tắt}

\begin{abstracts}
IoT (Internet of Things) hay Mạng lưới vạn vật kết nối internet là một thuật ngữ không còn gì lạ lẫm trên thế giới, bởi lợi ích của nó trong tất cả các lĩnh vực của cuộc sống. Internet of things hiện nay có thể được hình dung là bao gồm nhiều mạng, thiết bị kết nối với nhau dựa trên hàng loạt thông số kĩ thuật và tiêu chuẩn. RFID có thể trở thành cầu nối bằng cách cung cấp dữ liệu xác định đối tượng cụ thể tại một địa điểm tụ thể và thời gian chính xác. Thẻ tag đảm bảo thiết bị có một định danh để có thể được nhận dạng bởi các đặc tính duy nhất. Đó cũng chính là mức độ giao tiếp cơ bản đặc trưng của RFID và điều này hoàn toàn phù hợp với những gì mà IoT đòi hỏi.\\
Trong đó, Siêu cao tần thụ động (UHF – ultra high frequency) là một trong những công nghệ đang nổi lên như là những tiêu chuẩn có khả năng được ứng dụng rộng rãi trong thế giới IoT. Trong báo cáo này, chúng ta sẽ nghiên cứu sâu vào Công nghệ UHF RFID và Ứng dụng của nó trong cuộc sống.
\end{abstracts}

