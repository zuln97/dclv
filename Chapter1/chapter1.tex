\headerandfooterconfig
\graphicspath{Chapter1/Chapter1Figs/}
\chapter{Giới thiệu tổng quan đề tài}

\section{Ý tưởng nghiên cứu, phạm vi nghiên cứu và đối tượng nghiên cứu}
\subsection{Ý tưởng nghiên cứu}
Công nghệ RFID - Radio Frequency Identification (Nhận dạng tần số sóng vô tuyến) đang trở nên phổ biến với sự đa dạng các ứng dụng trong kỉ nguyên công nghệ 4.0. Với kiến trúc đơn giản, hệ thống RFID được sử dụng trong nhiều lĩnh vực: kiểm soát ra vào, nhận diện khách hàng, động vật, đồ vật,...Nhất là trong hệ thống quản lý hàng hóa, kho bãi, chuổi cung ứng,...
\Figure{htp}{12}{Chapter1/Chapter1Figs/fig1_2}{Kiến trúc cơ bản của một hệ thống RFID}
\label{ref{fig1_2}}
\\Trong đó, công nghệ UHF RFID có thể đọc đồng thời các đặc tính của các thẻ điện tử trường gần và trường xa. Ngoài loại thẻ kiểm soát truy cập thẻ ID truyền thống (kích thước thẻ tín dụng), công nghệ UHF RFID cung cấp nhiều loại thẻ điện tử để lựa chọn, có thể được cài đặt trong bất kỳ tài liệu nào và đang trong tình trạng tốt để theo dõi các mặt hàng, Container và phương tiện để cải thiện độ chính xác của hậu cần và ứng dụng theo dõi. Hiện nay, hầu hết các sản phẩm RFID UHF trên thị trường đều tập trung vào các ứng dụng như chuỗi cung ứng và quản lý hậu cần.

\subsection{Phạm vi nghiên cứu}
Vì sự giới hạn của thời gian và kiến thức, trong đề tài này, chúng ta sẽ nghiên cứu và ứng dụng một hệ thống UHF RFID đơn giản với đầy đủ kiến trúc của một hệ thống RFID thường thấy.

\begin{description}
	\item[Phần cứng] Các kiến thức về hệ thống UHF RFID, antenna, tag.
	\item[Phần mềm] Các kiến thức về đọc và xử lý dữ liệu RFID, xây dựng hệ thống Web Sever, lập trình Web, làm việc với cơ sở dữ liệu MongoDB.
\end{description}

\subsection{Đối tượng nghiên cứu}
\begin{itemize}
	\item Đọc thẻ tag.
	\item Xử lý dữ liệu trên tag.
	\item Cơ sở dữ liệu MongoDB.
	\item Web Service với NodeJS.
	\item Kết nối Web Server với CSDL MongoDB.
	\item Lập trình UI/UX Web.
\end{itemize}
\section{Tổng quan về kiến trúc của hệ thống}
\Figure{htp}{12}{Chapter1/Chapter1Figs/fig1_1}{Tổng quan về kiến trúc của hệ thống}
\label{ref{fig1_1}}
Như có thể thấy trong hình 1.2, các dữ liệu dùng để nhận dạng từ các tag riêng biệt sau khi được đọc bằng Reader RFID sẽ được xử lý và lưu vào hệ cơ sở dữ liệu MongoDB. Sau đó được kết nối với Web Sever NodeJs và hiển thị trên giao diện web cho người dùng quan sát.
